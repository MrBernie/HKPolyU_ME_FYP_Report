\chapter*{Conclusion}
\addcontentsline{toc}{chapter}{Conclusion}

% \section{Section}
% \subsection{Subsection}
% \subsubsection{Subsubsection}

The exploration of traditional SSL methods made significant progress. The experiments validating the feasibility of traditional algorithms were completed, although with possibilities to be improved. Both the TDoA method and the Beamforming method can roughly estimate the DoA of the sound source. For the future plan, XIAO Pengbo will be responsible for the improvement of traditional SSL methods, the improvement of the DL-based methods by utilizing the concept from traditional SSL methods.

The exploration of DL-based SSL methods and current testing of the DNN provides a promising result for estimating DoA using the computationally generated dataset. However, the performance of the DNN needs to be tested using the real-world recorded audio dataset. For the future plan, ZENG Bailin will be responsible for testing the feasibility of using this DNN structure for SSL by using audio recording in the laboratory and the anechoic chamber, solving any possible problems in this process and improving its performance.

The current progress of the mobile platform indicates the need for further development programs for the system. In the future, the code for mobile platform testing should be developed by WU Zhuoli. Further deployment of the traditional SSL algorithm or the DL-based algorithm on the mobile platform should be done in the coming months.

In the final overall conclusion, this FYP project is progressing as intended. Further development and exploration should be performed in the next several months to meet the final goal of this project.